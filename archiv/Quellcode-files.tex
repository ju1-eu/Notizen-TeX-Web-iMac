% ---------------------------------
% Quellcode 'code/' in Latex speichern. 
% 'archiv/Quellcode-files.tex' 
% HTML, Python, Bash, C, C++, TeX 
% ju 17-Jan-2021 Quellcode-files.tex
% ---------------------------------
%

\section{*}

%*.sh (\autoref{code:*-1}).% Referenz
%
\lstset{language=Bash}% HTML, Python, Bash, C, C++, TeX
\lstinputlisting[% anpassen
    caption={Quellcode in Bash: *.sh}, %label={code:*-1}
]{code/*.sh}% file

\newpage
\section{*}

%*.html (\autoref{code:*-1}).% Referenz
%
\lstset{language=HTML}% HTML, Python, Bash, C, C++, TeX
\lstinputlisting[% anpassen
    caption={Quellcode in HTML: *.html}, %label={code:*-1}
]{code/*.html}% file

\newpage
\section{hallowelt}

%hallowelt.c (\autoref{code:hallowelt-1}).% Referenz
%
\lstset{language=C}% HTML, Python, Bash, C, C++, TeX
\lstinputlisting[% anpassen
    caption={Quellcode in C: hallowelt.c}, %label={code:hallowelt-1}
]{code/hallowelt.c}% file

\newpage
\section{hallowelt-C++}

%hallowelt-C++.cpp (\autoref{code:hallowelt-C++-1}).% Referenz
%
\lstset{language=C++}% HTML, Python, Bash, C, C++, TeX
\lstinputlisting[% anpassen
    caption={Quellcode in C++: hallowelt-C++.cpp}, %label={code:hallowelt-C++-1}
]{code/hallowelt-C++.cpp}% file

\newpage
\section{*}

%*.py (\autoref{code:*-1}).% Referenz
%
\lstset{language=Python}% HTML, Python, Bash, C, C++, TeX
\lstinputlisting[% anpassen
    caption={Quellcode in Python: *.py}, %label={code:*-1}
]{code/*.py}% file

\newpage
\section{*}

%*.tex (\autoref{code:*-1}).% Referenz
%
\lstset{language=TeX}% HTML, Python, Bash, C, C++, TeX
\lstinputlisting[% anpassen
    caption={Quellcode in TeX: *.tex}, %label={code:*-1}
]{code/*.tex}% file

\newpage
